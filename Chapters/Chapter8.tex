\chapter{Conclusions and Future Work}

We have successfully made an open-source end-user system which provides the farmer with all the functionalities to analyse and improve the farm health. The drone is made autonomous for surveying/scanning the farm. Image stitching is then done using OpenDroneMap library. Application of NDVI combined with SoftMax regression model gave us an accuracy of about 70\% in detecting critical areas on the farm. Analysis of the images taken by the smartphone of the critical areas on the farm is also done using Google’s Inception V-3 model receiving an accuracy of about 99\%. An easy to use web and android based application is also made to ease user’s access. Thus a low cost system for precision farming has been built using all open-source software and libraries with minimal proprietary cost. Since, we have made our own labelled dataset using clustering and trained our machine learning model on this data this creates inaccuracy in results which can be improved once we have a better labelled dataset. We can also make use of more VI methods like Soil Adjusted Vegetation Index (SAVI), Enhanced NDVI (ENDVI), etc. to improve the accuracy of our machine learning model. Also the limitations in the deep learning model can be removed once we have an improved dataset which includes pictures taken from various angles, having varied backgrounds, etc. so that the model becomes more feasible for practical scenarios. The application just now runs on localhost and needs to be deployed on a server.  